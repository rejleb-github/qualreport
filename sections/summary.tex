\section{Project Summary}
\subsection{Overview} 
\begin{enumerate}
  \item Provide a unified framework for fast MCMC techniques which uses both TTV and RV which is used to do bayesian inference on the parameters, this pipeline can then be used for future missions.
  \item New and Hybrid MCMC will be investiged. Main focus on adding Langevin based MCMCs (This is novel, paper in works on Simple SMALA).
  \item Analyze data from various exoplanet search projects to detect missed TTV (or refine RV?). The main dataset to be reanalysed will be Kepler but the total set of missions/projects which could provide data include: \memorl{Red/color to highlight main dataset of interest?}
      \begin{enumerate}
      \item Ground Based
      		\begin{enumerate}
      		\item HARPS
      		\item HATNet Projects
      		\item SuperWASP
      		\end{enumerate}
      \item Space Missions
      		\begin{enumerate}
            \item COROT
      		\item \textcolor{red}{Kepler/K2}
           	\item CHEOPS
      		\item TESS
            \item JWST
      		\end{enumerate}
      \end{enumerate}
  \item This is how we complete our objective: improve inferences on planet population detectable from TTV and RV This is useful for doing dynamic studies and provides insight or clues about potential planet formation mechanism for the target population.
  \item I can then do an in-depth investigation on select systems which feature unusual transit or dynamical features.
  \item Enhance the open exoplanet catalogue by providing MCMC dataset results, allowing others to perform desired Bayesian analysis with minimal computational cost. Especially good from a reproducibility perspective, hence scientific merit.
  \item Additionally, these inferences on orbital elements give us a collection of posteriors which can be used to provide universal MCMC priors for this target population as we aim to process a considerable volume of systems.
\end{enumerate}


\subsection{Intellectual Merit} 
Expand on points 4 and 5. Apprx. 2 paragraphs.
\subsection{Broader Impacts of the Proposed Work} 
Expand on points 6 and 7. Apprx. 2 paragraphs.
