\section{Project Summary}
\subsection{Overview}
While there are many methods for observing exoplanets, the ones considered in this work are transit timing variations (TTV) and radial velocity (RV). 
TTVs are observed when at least one planet in a multiplanet system transit over a star. 
In a 2-body setting the transits occur with a perfect period, but when other bodies interact with the transiting planet, we obtain variations in the transit time.
By measuring these subtle variations in timing, we can infer the number of planets present in the system and their orbital parameters.
Radial velocity surveys are done by measuring the Doppler shift experienced by a star as one or more exoplanets dynamically interact with it.


Both of these discovery methods use analytical, but more often, N-body simulations to carry out inference on the orbital parameters.
This project aims to create a unified framework to analyze TTV and RV datasets together, and perform Bayesian inference on the orbital parameters.
This inference is typically done with Monte Carlo Markov Chains (MCMC).
This approach is considered by other researchers in the field [reference].
This pipeline will be coupled with \reb and make use of its variational equations to open up research opportunities which have not been previously considered.
The main physics objective of the project is to discover and reanalyze planetary systems in order to do a statistical analysis on the population detectable by TTV and RV in a single unified framework 
Some studies have benefited of additional data and revisits [reference Trappist-1 mass change].
This sort of analysis will allow us to probe dynamical constraints on planetary systems and provide clues on their formation process in a self-consistent manner.
As supplementary objectives, we aim to develop faster MCMC methods with the use of second order variational equations.
We will also explore various hybrid MCMC methods which will be optimized for this problem.
This analysis will also produce good Bayesian priors for future analysis of systems discovered by TTV or RV.


We plan on making the statistical data generated from the inference available publicly on the Open Exoplanet Catalog in the interest of scientific repeatability.
This study can utilize data from the many sources, including future missions which may release new data in the scope of this PhD program. 
The datasets of interest are listed below.
\begin{enumerate}
      \item Ground Based
		\begin{enumerate}
      		\item HARPS
      		\item HATNet Projects
      		\item SuperWASP
      		\end{enumerate}
      \item Space Missions
      		\begin{enumerate}
            	\item COROT
      		\item Kepler/K2
           	\item CHEOPS
      		\item TESS
           	 \item JWST
      		\end{enumerate}
 \end{enumerate}
We will operate mostly on the Kepler data in hopes of finding previously missed systems with TTVs, along with the new incoming data from the JWST mission.

\subsection{Intellectual Merit} 
By refining available data and combining it with new data we may be able to make strong dynamical constraints on the planet formation process. 
If successful we could provide strong evidence for or against planet formation models such as the Nice model.
When looking at the new incoming data, we will also be discovering new planets which will lead to a better understanding of the statistical ensemble of planets detectable by TTV and RV.
By considering select systems which feature unusual transits or dynamical behavior we can gain insights into what sorts of stability criterion are most important `in the wild'.

\subsection{Broader Impacts of the Proposed Work} 
By performing inference on orbital parameters in a self-consistent, open, and reproducible approach, it becomes much easier to analyze the global statistical properties of planetary ensemble.
With these large scale statistics and database, we can provide priors which can be easily updated for future inference of orbital parameters.
In terms of broadest and long term impacts, given a better understanding of planetary formation theory, this will help direct searches for life in the universe.





